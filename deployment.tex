%This work is licensed under the Creative Commons Attribution-NonCommercial-NoDerivs 3.0 United States License. To view a copy of this license, visit http://creativecommons.org/licenses/by-nc-nd/3.0/us/ or send a letter to Creative Commons, 444 Castro Street, Suite 900, Mountain View, California, 94041, USA.

\section{Deployment}

Mojolicious applications are either a one-file Lite app
%TODO link to lite app
or else a well-structured multi-file Full app.
%TODO link to full app
In either case, there will be one script file which will act as the deployment point for your application.
By convention, you will often see this file called \verb!app.pl!, but it may be named whatever you choose.
%TODO add app.pl to conventions list
Mojolicious comes with an example script, simply called \verb!mojo!, which may be used to test the daemon or other script-level actions.

A Mojolicious app may be deployed in many ways.
Your choice of which deployment mechanism to choose could hinge on of several factors.
What is the platform and capability of your server?
What Mojolicious features or PSGI middleware does your application need?

\subsection{Mojolicious' Built-in Servers}

Mojolicious comes with several servers as part of its distribution.
The most important of these is the \emph{daemon}.
All of the other built-in servers are just extensions of the daemon.

\subsubsection{daemon}

The daemon is a highly capable real-time server, built on top of the Mojo::IOLoop event loop system.
%TODO link to chapter on Mojo::IOLoop
In order to use Mojolicious' real-time features, like WebSockets for example,
%TODO link to websocket chapter
you will need to use one of the built-in servers.

To start the daemon, simply pass \verb!daemon! as a command line argument to your script (i.e. \verb!app.pl daemon!). 
This will start your application using the daemon and bind it to port 3000.
The daemon has serveral options, which may be displayed by running \verb!app.pl help daemon!.
See them now by running \verb!mojo help daemon!.

\subsubsection{morbo}

While the daemon is very useful, it is only the lowest level server that Mojolicious provides.
When writing your application, you might find it hand to use the development server \verb!morbo!.
Morbo is a wrapper around the daemon, with all of its same capabilites, but with a very helpful development feature:
it restarts itself automatically whenever one of your application's files change!

Unlike the daemon, which is started as an argument to your application script, morbo is its own script.
To start your application using morbo, pass the script as an argument to the \verb!morbo! command (i.e. \verb!morbo app.pl!).

Morbo allows some, but not necessarily all, of the same options as daemon, and provides some of its own.
To see them all, run \verb!morbo --help!.
The most useful of these options is the \verb!--watch! or \verb!-w! flag.
By default, morbo will watch all of the files that it can understand to be part of your application,
still there may be times when you have more files whose change should trigger a restart of the server.
Use the watch flag to specify more files that morbo should watch.

\subsubsection{hypnotoad}

ALL GLORY TO THE HYPNOTOAD!!!

\verb!hypotoad! is Mojolicious' top performer!
It is a highly tuned, preforking server, which can handle whatever you throw at it.
Except Windows, that is.
It screams on Linux (and Mac, but who has a Mac server?).

Like \verb!morbo!, you call \verb!hypnotoad! with your application as an argument.
Unlike \verb!morbo! most of its options are not given on the command line, but as configuration parameters.
%TODO link to config plugin
This is done to accomodate its other major feature: hot reloading.

When you have an application running, you don't want to have to take it down to do upgrades.
But upgrades happen.
Eventually you will want to add features to your app, or want new Mojolicious features, or even want to upgrade Perl.
With any other deployment option, you will need to take your site down, at least briefly, to reload.
%TODO check any vs most
Not so with \verb!hypnotoad!.

When using \verb!hypnotoad!, simply run the same command you used to start it, and it will hot-readload.
You won't even lose the current connections!
Workers will drop after they finish serving their current clients, and then will be replaced by new ones using the upgrades.

To stop the application, run the same command as you did to start your app, but add the \verb!--stop! or \verb!-s! option.
It will even gracefully stop, meaning it will wait to finish serving the current clients.
Alternately, if you would like to prevent \verb!hypnotoad! from daemonizing, 
keeping it running in the foreground, pass the \verb!--foreground! or \verb!-f! option.


\subsection{PSGI Servers}

PSGI has been an exciting development in Perl web application technology.
It is an abstraction layer which allows web applications to be built in a deployment agnostic way.
Most major Perl frameworks have ported to only act in a PSGI compliant manner, and for most frameworks, this is enough.

Unfortunately, PSGI does not define a real-time API and thus Mojolicious is not a pure PSGI framework.
That said, it can act in a PSGI compliant manner, and will do so when deploying via a PSGI server.
To see that, we can run our app using the reference implementation called \verb!Plack!, and its server \verb!plackup!, by running \verb!plackup myapp.pl!.
To try the Mojolicious built-in app under \verb!plackup!, try \verb!plackup $(which mojo)!.

While you lose the real-time features of Mojolicious when using a PSGI server, you gain the use of the plack middlewares.
%TODO include middleware example
Read more about middlewares at the documentation here: \url{https://metacpan.org/pod/Plack::Middleware}.

\subsection{CGI}

Mojolicious applications will even run as CGI scripts.
Simply place the script file in your properly configured \verb!cgi-bin! and off you go!
As with PSGI, real-time features will not work, but in certain constrained situations, where CGI is all you have, Mojolicious won't let you down!
Of course, part of the reason PSGI is so attractive now is that configuring your server for CGI can be a pain ---
and for that reason, this book will not cover CGI usage whatsoever.
There are plenty of online resources that will help you find your way.

\subsection{Via the command line}

Finally, and though this will be covered in more detail later, your app be directly invoked on the command line with the \verb!get! command.
If your application defines the \verb!/path/page.html! resource, you can simply run \verb!./myapp.pl get /path/page.html! to see the output on your terminal.
This is great for debugging!
The \verb!get! command is actually even more powerful than this, but let's save that for later.
%TODO include link to get command


