\section{A Brief Note About Mojolicious::Lite} \label{sec:lite_not_magical}

%XXX consider making an architecture section

When designing a Mojolicious application, authors usually follow the accepted Model-View-Controller architecture.
%TODO external links to MVC
The model layer is intentionally unspecified.
Perl, and its massive CPAN (see \url{http://metacpan.org} for more) have many options for database interaction.
%TODO link could be a footnote
Mojolicious is agnostic to the Model layer, though some suggestions for useful modules are \cpan{DBIx::Connector} for connection management or \cpan{DBIx::Class} as an ORM.
%TODO ext link about ORM
Also worthy of a mention is the Mojolicious spin-off project \cpan{Mango}, a non-blocking interface to MongoDB.

Mojolicious applications typically are a well structured set of files, defining Perl classes, templates, tests and other supporting files.
These classes represent your application and its controllers, in addition to any model logic that you might include.
%TODO links to both here
While using this full structure is highly encouraged for production, for fast prototyping, and especially for presenting examples, this structure becomes cumbersome.

Luckily, Mojolicious comes with tiny wrapper layer called Mojolicious::Lite.
This module provides a functional dsl wrapping application methods, router methods and helpers --- all of which will be discussed in due course --- so that simple applications can be defined in a single script.

\subsection{The Lite DSL}

With the Mojolicious::Lite sugar, a ``hello world'' app looks something like this:
\begin{mojolite}
#!/usr/bin/env perl
use Mojolicious::Lite;

any '/' => sub { shift->render( text => 'hello world' ) };

app->start;
\end{mojolite}

This syntax will be used throught this text for demonstrating Mojolicious concepts.
Interestingly, many users seem to have trouble translating the Mojolicious::Lite dsl into the full object-oriented equivalent.
This is unfortunate, since the translation is very simple.
Indeed as of the time of this writing, the module which defines the dsl (\verb!Mojolicious/Lite.pm!) is only 59 lines long!

The new kewords \lstmojolite!any!, \lstmojolite!get!, \lstmojolite!options!, \lstmojolite!patch!, 
\lstmojolite!post!, \lstmojolite!put!, and \lstmojolite!websocket! define routes using the router method of the same name. 
%TODO link to router methods
The keyword \lstmojolite!del! does the same, except it uses the router method \lstperl!delete!.
In the example above, the router method \lstperl!any! is used (via the \lstmojolite!any! keyword) 
to define the root path and when requested it will render the text \lstperl!'hello world'!.
The subroutine reference given to the route definition is simply used as a controller method; 
this is called a hybrid route and is available even in the full syntax, for ease of translation.

The new keyword \lstmojolite!app! is used to get a reference to application object, 
while the keywords \lstmojolite!helper!, \lstmojolite!hook!, and \lstmojolite!plugin! call the application methods of the same name.
%TODO link to application methods
In the above example, \lstmojolite!app->start! is used to call the \lstperl!start! method on the application object.

Only two of the additional keywords, \lstmojolite!under! and \lstmojolite!group!, do not translate directly to full-app methods.
The \lstmojolite!under! keyword creates a bridge using the router method of the same name, 
then uses that bridge for any route definition following it.
The \lstmojolite!group! keyword can be used to scope that change, and therefore has is not needed in full apps, 
where the bridge object itself is used to create dependent routes.
In this text, when these keywords are used in a lite app context, translation to full syntax will be provided.

The author does not expect the reader to understand concepts in this explaination, yet.
What is important to realize, however, is that the Mojolicious::Lite syntax/dsl is not magical.
The fact that this text and the official documentation use lite syntax frequently is only due to its concise nature.
These examples should not discourage users from trying the full object-oriented syntax and structure.

\subsection{Strict, Warnings and Friends}

Finally, a reader familiar with Perl might notice that the important declarations \lstperl!use strict;! 
and \lstperl!use warnings;! are not in the ``hello world'' example above.
If you noticed that, then, well spotted!
These commands are highly encouraged for all Perl scripts, so why are they absent?

Any script which imports \lstperl!Mojolicious::Lite! or \lstperl!Mojo::Base! (and all templates) will automatically import 
\lstperl!strict!, \lstperl!warnings!, \lstperl!utf8!.
In addition, all of the Perl 5.10 features are automatically imported, since Mojolcious itself requires at least that version.
This includes the \lstperl!state! keyword, which is used regularly in these examples as well as the Mojolicious code base.

To use this import in other scripts, simply add \lstperl!use Mojo::Base -strict;! in place of the normal incantations.
